\documentclass{article}
\usepackage{minted}
\renewcommand{\familydefault}{\sfdefault}
\renewcommand{\DeleteFile}[2][]{}

\title{Java Control Statements}
\date{2021-02-23}
\author{Liv}

\begin{document}
    \maketitle
    \pagenumbering{arabic}

    \section{Questions and Examples}

    \subsection{The most basic control flow statement supported by the Java programming language is the \_\_\_\_\_ statement.}
    The if-then statement.

    \subsection{The \_\_\_\_\_ statement allows for any number of possible execution paths.}
    The switch statement does.

    \subsection{The \_\_\_\_\_ statement is similar to the while statement, but evaluates it's expression at the \_\_\_\_\_ of the loop.}
    The do-while evaluates it's expression at the bottom of the loop.

    \subsection{How do you write an infinite loop using the for statement?}
    \begin{minted}{java}
        for (int i = 0; i > 0; i++) {
            System.out.println("This is an infinite loop");
        }
        for ( ; ; ) {
            System.out.println("This is also an infinite loop");
        }
    \end{minted}

    \subsection{How do you write an infinite loop using the while statement?}
    \begin{minted}{java}
        while (true) {
            System.out.println("This is an infinite loop");
        }
    \end{minted}

    \subsection{Consider the following code snippet.}
    \subsubsection{What output do you think the code will produce if aNumber is 3?}
    \begin{minted}{java}
    int aNumber = 3;

    if (aNumber >= 0) { // This block is entered because 3 is >= 0
        if (aNumber == 0) { 
            System.out.println("first string"); // does not print
        } else { 
            System.out.println("second string"); // prints
        }
    } 
    System.out.println("third string"); // prints
    \end{minted}

    \subsubsection{Write a test program containing the previous code snippet, make aNumber 3. What is the output of the program? Is it what you predicted? Explain why the output is what it is; in other words, what is the control flow for the code snippet?}

    \begin{minted}{java}
    public class NumberTest {
        public static void main(String[] args) {
            int aNumber = 3;

            if (aNumber >= 0) { // This block is entered because 3 is >= 0
                if (aNumber == 0) { 
                    System.out.println("first string"); // does not print
                } else { 
                    System.out.println("second string"); // prints
                }
            } 
            System.out.println("third string"); // prints
        }
    }
    \end{minted}

    \paragraph{}
    The output of the program is as follows, as predicted by the code comments. I explained the control flow in the comments, I feel it is clear enough.

    \begin{minted}{java}
    second string
    third string
    \end{minted}

    \newpage
    \subsubsection{Using only spaces and line breaks, reformat the code snippet to make the control flow easier to understand.}
    
    I feel this is as clear as you will get with only spaces and linebreaks.
    
    \begin{minted}{java}
    if (aNumber >= 0)
        if (aNumber == 0) System.out.println("first string");
        else System.out.println("second string");
    
    System.out.println("third string");
    \end{minted}

    \subsubsection{Use braces, { and }, to further clarify the code.}
    I have already done this in examples 1.6.1 and 1.6.2.

    \subsection{Many centuries ago in India a very smart man is said to have invented the game of chess.}
    I feel like I'm too tired to do this but oh well. I'm also doing this without an IDE or testing so I will be surprised if it works.

    \begin{minted}{java}
    public class RiceFarmer {
        public static void main(String[] args) {
            int rice = 1;
            
            for (i = 0; i < 64; i++) {
                rice *= 2;
            }
            
            System.out.println("Total rice grains are: " + rice);
        }
    }
    \end{minted}

    \subsection{Write a program that simulates rolling a pair of dice.}
    This is dumb.
    \begin{minted}{java}
    public class DiceRoller {
        public static void main(String[] args) {
            int roll1 = (int) (Math.random()*6) + 1;
            int roll2 = (int) (Math.random()*6) + 1;
            
            System.out.println("The first die comes up " + roll1);
            System.out.println("The second die comes up " + roll2);
            System.out.println("Your total roll is " + (roll1 + roll2));
        }
    }
    \end{minted}

    \subsection{How many times do you have to roll a pair of dice before they come up snake eyes?}

    \begin{minted}{java}
    public class DiceRoller {
        public static void main(String[] args) {
            int attempts = 0;
            Boolean snakeEyes = false;

            do {
                int roll1 = (int) (Math.random()*6) + 1;
                int roll2 = (int) (Math.random()*6) + 1;

                attempts++;
                if (roll1 == 0 && roll2 == 0) {
                    snakeEyes = true;
                }
            } while (!snakeEyes)
            
            System.out.println("Found snake eyes after " + attempts + "attempts");
        }
    }
    \end{minted}

    \subsection{Write a program to extract each digit from an int, in the reverse order.}
    This could probably be more efficient if I used an IDE honestly, but I think it works.

    \begin{minted}{java}
    public class DigitExtractor {
        public static void main(String[] args) {
            int in = 1542;
            String reversed = "";

            for (int i = 0; i < String.valueOf(in).length(); i++) {
                reversed += String.valueOf((int) (in / (Math.pow(10, i))) % 10);
            }
            
            System.out.println(reversed);
        }
    }
    \end{minted}

\end{document}